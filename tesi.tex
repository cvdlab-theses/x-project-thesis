\documentclass[a4paper,11pt,oneside,titlepage,openright]{book}
\usepackage[italian]{babel}              % italiano
%\usepackage[latin1]{inputenc}   % accenti
\usepackage[utf8]{inputenc}   % accenti
\usepackage{indentfirst}
\usepackage[dvips]{graphicx}    % per includere i grafici
\usepackage{amssymb}            % per i simboli matematici
\usepackage{amsmath}            %            "
\usepackage{latexsym}           %            "
\usepackage{amsthm}             %            "
\usepackage{amsfonts}
\usepackage{lettrine}           % per il capolettera dell'introduzione
\usepackage{epsfig}             % per includere le figure
\usepackage{hyperref}
\usepackage{graphicx}
\usepackage{color}
\usepackage{courier}
\usepackage{caption}
\usepackage{mdwlist}
\usepackage{listings}
\usepackage[T1]{fontenc}
\usepackage{booktabs}
\usepackage{longtable}
\usepackage{pdflscape}
\usepackage{rotating}
\usepackage[babel]{csquotes}
\usepackage{epigraph}
\usepackage{tabularx}
\usepackage[style=numeric-comp,useprefix,hyperref,backend=bibtex]{biblatex}
\bibliography{tex_files/bibliografia/bibliografia}


\pagestyle{headings}

\lstdefinelanguage{JavaScript}{
  keywords={typeof, new, true, false, catch, function, return, null, catch, switch, var, if, in, while, do, else, case, break},
  keywordstyle=\color{blue}\bfseries,
  ndkeywords={class, export, boolean, throw, implements, import, this},
  ndkeywordstyle=\color{darkgray}\bfseries,
  identifierstyle=\color{black},
  sensitive=false,
  comment=[l]{//},
  morecomment=[s]{/*}{*/},
  commentstyle=\color{purple}\ttfamily,
  stringstyle=\color{red}\ttfamily,
  morestring=[b]',
  morestring=[b]"
}


\DeclareCaptionFont{white}{\color{white}}
\DeclareCaptionFormat{listing}{\colorbox[cmyk]{0.43, 0.35, 0.35,0.01}{\parbox{\textwidth}{\hspace{15pt}#1#2#3}}}
\captionsetup[lstlisting]{format=listing,labelfont=white,textfont=white, singlelinecheck=false, margin=0pt, font={bf,footnotesize}}


\definecolor{lightgray}{rgb}{.9,.9,.9}
\definecolor{darkgray}{rgb}{.4,.4,.4}
\definecolor{purple}{rgb}{0.65, 0.12, 0.82}
\definecolor{Brown}{cmyk}{0,0.81,1,0.60}
\definecolor{OliveGreen}{cmyk}{0.64,0,0.95,0.40}
\definecolor{CadetBlue}{cmyk}{0.62,0.57,0.23,0}
\definecolor{lightlightgray}{gray}{0.9}





% per l'interlinea
\linespread{1.5}

\newcommand{\minisizeurl}[1]{\footnotesize #1 \normalsize}

\newcommand{\terminale}{
    \lstset{language={}}
}

\newcommand{\javascript}{
    \lstset{
    language=JavaScript,
    backgroundcolor=\color{lightgray},
    extendedchars=true,
    basicstyle=\footnotesize\ttfamily,
    showstringspaces=false,
    showspaces=false,
%    numbers=left,
%    numberstyle=\footnotesize,
%    numbersep=9pt,
    tabsize=2,
    breaklines=true,
    showtabs=false,
    captionpos=b
   }
}

\newcommand{\html}{
    \lstset{
    language=HTML,                          % Code langugage
    basicstyle=\ttfamily\footnotesize,       % Code font, Examples: \footnotesize, \ttfamily
    keywordstyle=\color{OliveGreen},        % Keywords font ('*' = uppercase)
    commentstyle=\color{gray},              % Comments font
    %numbers=left,                           % Line nums position
    %numberstyle=\tiny,                      % Line-numbers fonts
    stepnumber=1,                           % Step between two line-numbers
    numbersep=5pt,                          % How far are line-numbers from code
    backgroundcolor=\color{lightgray}, % Choose background color
    frame=none,                             % A frame around the code
    tabsize=2,                              % Default tab size
    captionpos=b,                           % Caption-position = bottom
    breaklines=true,                        % Automatic line breaking?
    breakatwhitespace=false,                % Automatic breaks only at whitespace?
    showspaces=false,                       % Dont make spaces visible
    showtabs=false,                         % Dont make tabls visible
    stringstyle=\color{red}\ttfamily,
    xleftmargin=17pt,
    framexleftmargin=17pt,
    framexrightmargin=5pt,
    framexbottommargin=4pt,
    }
}

\newcommand{\xml}{
  \lstset{
    language=XML,                          % Code langugage
    basicstyle=\ttfamily\footnotesize,       % Code font, Examples: \footnotesize, \ttfamily
    keywordstyle=\color{OliveGreen},        % Keywords font ('*' = uppercase)
    commentstyle=\color{gray},              % Comments font
    %numbers=left,                          % Line nums position
    %numberstyle=\tiny,                     % Line-numbers fonts
    stepnumber=1,                           % Step between two line-numbers
    numbersep=5pt,                          % How far are line-numbers from code
    backgroundcolor=\color{lightgray},      % Choose background color
    frame=none,                             % A frame around the code
    tabsize=2,                              % Default tab size
    captionpos=b,                           % Caption-position = bottom
    breaklines=true,                        % Automatic line breaking?
    breakatwhitespace=false,                % Automatic breaks only at whitespace?
    showspaces=false,                       % Dont make spaces visible
    showtabs=false,                         % Dont make tabls visible
    stringstyle=\color{red}\ttfamily,
    xleftmargin=17pt,
    framexleftmargin=17pt,
    framexrightmargin=5pt,
    framexbottommargin=4pt,
    }
}

%% per le estensioni
\newcommand{\ext}{png}

%% per le definizioni
\newcommand{\definizione}[1]{{\em {\bfseries #1}}}

%%VisPro maiuscoletto
\newcommand{\vp}{V{\textsc is}P{\textsc ro} }
\newcommand{\vpunspaced}{V{\textsc is}P{\textsc ro}}


%% comandi per la customizzazione della stringa dei riferimenti
\newcommand{\vedi}[1]{(si veda #1)}

\hyphenation{sil-la-ba-zio-ne}                   % per la corretta sillabazione delle parole
\hyphenation{del-le} \hyphenation{pa-ro-le}

% per gli esempi
\theoremstyle{plain}
\newtheorem{thm}{Teorema}[section]
\theoremstyle{definition}
\newtheorem{defn}{Definizione}[chapter]
\newtheorem{ex}{Esempio}[chapter]               % [CHAPTER] per la numerazione degli esempi
\theoremstyle{remark}
\newtheorem{codifica}{Codifica}

%%% DOCUMENT %%%

\begin{document}

\chapter{X-Project}
\label{cha:chapter_4}

This chapter presents the core of the thesis project: X-Project.

The first section provides a project's overview, giving reasons of development, listing benefits and functions. Second section there's a brief explanation of the reasons that are behind the name. The third section presents a practical example of some of X-Project functions. Fourth section shows the X-Project architectural stack and the reasons why these technologies have been chosen. In the fifth section is presented the development methodology that has been thought for the project. Sixth section presents a set of elements: most important practical part of the project.

\section{Section 1}
\label{sec:XPR_section_1}

In this section...

\section{Section 2}
\label{sec:XPR_section_2}

In this section...

\paragraph{}
In this chapter the main part of thesis project has been described: X-Project.
First of all, it an overview of the project has been provided, later, the name explaination it has been introduced. In the last sections, X-Project architecture functionality have been introduced.


 %Frontespizio comune
% \part{x-project}
	\label{seconda}
	\lettrine[lines=1]{I}{l browser} ...

	\chapter{Architettura}
		\label{cha:architettura}
		\input{tex_files/parte_2/architettura/intro}
\label{sec:architettura_intro}

\section{Architettura}
	\label{sec:architettura}
	\input{tex_files/parte_2/architettura/architettura}


	\chapter{Media management}
		\label{cha:media_management}
		\input{tex_files/parte_2/media_management/intro}
\label{sec:media_management_intro}

\section{Media management}
	\label{sec:media_management}
	\input{tex_files/parte_2/media_management/media_management}



	% \chapter{User management}
	% 	\label{cha:user_management}
	% 	\input{tex_files/parte_2/user_management/intro}
\label{sec:user_management_intro}

\section{User management}
	\label{sec:user_management}
	\input{tex_files/parte_2/user_management/user_management}


	\chapter{Caso d'uso}
		\label{cha:caso_uso}
		\input{tex_files/parte_2/caso_uso/intro}
\label{cha:caso_uso_intro}

\section{Caso d'uso}
	\label{sec:caso_uso}
	\input{tex_files/parte_2/caso_uso/caso_uso}
 %Frontespizio D'Amelio
%%%% TITOLO %%%

\baselineskip=12mm \thispagestyle{empty}
%\begin{figure}                         % modo alternativo di includere le figure
%\centering
%\includegraphics[width=2cm]{logo.eps}
%\end{figure}
\begin{figure}
\begin{center}
~~\centerline {\psfig{file=images/logo/logo.png,width=5cm}}
\end{center}
\end{figure}
\vskip 1.40cm
\begin{center}
\large
Università degli Studi \textit{``Roma Tre''}\\
Facoltà di Ingegneria\\
Corso di Laurea Magistrale in Ingegneria Informatica\\
\end{center}
\vskip 1cm \Large
\begin{center}
Tesi di laurea magistrale
\end{center}
\vskip 1cm \baselineskip=10mm
\begin{center}
%TITOLO               \\
%skip 1cm
\textbf{{\em x-project - HTML5 Web Components CMS framework}}\\
\end{center}
\vskip .5cm
\begin{center}
Laureando\\
{Tiziano Sperati}\\
\end{center}
\vskip 1cm
\begin{center}
\large{
\begin{tabularx}{\textwidth}{>{\centering\arraybackslash}X >{\centering\arraybackslash}X}
  Relatore & Correlatori\\
  Prof. Alberto Paoluzzi & Dott.~Enrico Marino, Dott.~Federico Spini\\
\end{tabularx}
}
\end{center}
\vskip 1.5cm \centerline{Anno Accademico 2014/2015} \normalsize
 %Frontespizio Sperati

\newpage
%\input{tex_files/frontespizio/dedica_damelio} %Dedica D'Amelio
%\newpage\null\thispagestyle{empty}\newpage %pagina bianca dopo il mio frontespizio
\thispagestyle{empty}
\begin{flushright}
\null\vspace{\stretch{1}}
{\em Ad Eraclito, bla bla, bla bla bla} \\
{\em bla bla bla bla bla bla} \\
{\em bla bla bla}
\vspace{\stretch{2}}\null
\end{flushright} %Dedica Sperati

\frontmatter
\tableofcontents
\listoffigures

%\chapter{Ringraziamenti}
\label{cha:ringraziamenti}

Il presente lavoro non sarebbe stato possibile senza il supporto di ... 

 %Ringraziamenti comune
%\part{x-project}
	\label{seconda}
	\lettrine[lines=1]{I}{l browser} ...

	\chapter{Architettura}
		\label{cha:architettura}
		\input{tex_files/parte_2/architettura/intro}
\label{sec:architettura_intro}

\section{Architettura}
	\label{sec:architettura}
	\input{tex_files/parte_2/architettura/architettura}


	\chapter{Media management}
		\label{cha:media_management}
		\input{tex_files/parte_2/media_management/intro}
\label{sec:media_management_intro}

\section{Media management}
	\label{sec:media_management}
	\input{tex_files/parte_2/media_management/media_management}



	% \chapter{User management}
	% 	\label{cha:user_management}
	% 	\input{tex_files/parte_2/user_management/intro}
\label{sec:user_management_intro}

\section{User management}
	\label{sec:user_management}
	\input{tex_files/parte_2/user_management/user_management}


	\chapter{Caso d'uso}
		\label{cha:caso_uso}
		\input{tex_files/parte_2/caso_uso/intro}
\label{cha:caso_uso_intro}

\section{Caso d'uso}
	\label{sec:caso_uso}
	\input{tex_files/parte_2/caso_uso/caso_uso}
 %Ringraziamenti D'Amelio
%\chapter{Ringraziamenti}
\label{cha:ringraziamenti}

Ringrazio... %Ringraziamenti Sperati

\chapter{Introduzione}
\label{cha:introduzione}

Introduzione...

\input{tex_files/introduzione/intro_damelio} %Componente Introduzione D'Amelio

\input{tex_files/introduzione/intro_sperati} %Componente Introduzione Sperati

\input{tex_files/introduzione/ending}

\mainmatter

%%% PARTE PRIMA %%%

\part{Il contesto e le tecnologie}
	\label{part:prima}

	\chapter{CMS}
		\label{cha:cms}
		\input{tex_files/parte_1/cms/intro}
\label{sec:cms_intro}

\section{Statistiche CMS}
	\label{sec:cms_statistiche}
	\input{tex_files/parte_1/cms/statistiche}


	\chapter{Tecnologie abilitanti}
		\label{cha:tecnologie}
		\input{tex_files/parte_1/tecnologie/intro}
\label{sec:tecnologie_intro}

\section{Single Page Applications}
	\label{sec:spa}
	\input{tex_files/parte_1/tecnologie/spa}



\chapter{X-Project}
\label{cha:chapter_4}

This chapter presents the core of the thesis project: X-Project.

The first section provides a project's overview, giving reasons of development, listing benefits and functions. Second section there's a brief explanation of the reasons that are behind the name. The third section presents a practical example of some of X-Project functions. Fourth section shows the X-Project architectural stack and the reasons why these technologies have been chosen. In the fifth section is presented the development methodology that has been thought for the project. Sixth section presents a set of elements: most important practical part of the project.

\section{Section 1}
\label{sec:XPR_section_1}

In this section...

\section{Section 2}
\label{sec:XPR_section_2}

In this section...

\paragraph{}
In this chapter the main part of thesis project has been described: X-Project.
First of all, it an overview of the project has been provided, later, the name explaination it has been introduced. In the last sections, X-Project architecture functionality have been introduced.


 %Parte2 comune
%\part{x-project}
	\label{seconda}
	\lettrine[lines=1]{I}{l browser} ...

	\chapter{Architettura}
		\label{cha:architettura}
		\input{tex_files/parte_2/architettura/intro}
\label{sec:architettura_intro}

\section{Architettura}
	\label{sec:architettura}
	\input{tex_files/parte_2/architettura/architettura}


	\chapter{Media management}
		\label{cha:media_management}
		\input{tex_files/parte_2/media_management/intro}
\label{sec:media_management_intro}

\section{Media management}
	\label{sec:media_management}
	\input{tex_files/parte_2/media_management/media_management}



	% \chapter{User management}
	% 	\label{cha:user_management}
	% 	\input{tex_files/parte_2/user_management/intro}
\label{sec:user_management_intro}

\section{User management}
	\label{sec:user_management}
	\input{tex_files/parte_2/user_management/user_management}


	\chapter{Caso d'uso}
		\label{cha:caso_uso}
		\input{tex_files/parte_2/caso_uso/intro}
\label{cha:caso_uso_intro}

\section{Caso d'uso}
	\label{sec:caso_uso}
	\input{tex_files/parte_2/caso_uso/caso_uso}
 %Parte2 D'Amelio
%%%% TITOLO %%%

\baselineskip=12mm \thispagestyle{empty}
%\begin{figure}                         % modo alternativo di includere le figure
%\centering
%\includegraphics[width=2cm]{logo.eps}
%\end{figure}
\begin{figure}
\begin{center}
~~\centerline {\psfig{file=images/logo/logo.png,width=5cm}}
\end{center}
\end{figure}
\vskip 1.40cm
\begin{center}
\large
Università degli Studi \textit{``Roma Tre''}\\
Facoltà di Ingegneria\\
Corso di Laurea Magistrale in Ingegneria Informatica\\
\end{center}
\vskip 1cm \Large
\begin{center}
Tesi di laurea magistrale
\end{center}
\vskip 1cm \baselineskip=10mm
\begin{center}
%TITOLO               \\
%skip 1cm
\textbf{{\em x-project - HTML5 Web Components CMS framework}}\\
\end{center}
\vskip .5cm
\begin{center}
Laureando\\
{Tiziano Sperati}\\
\end{center}
\vskip 1cm
\begin{center}
\large{
\begin{tabularx}{\textwidth}{>{\centering\arraybackslash}X >{\centering\arraybackslash}X}
  Relatore & Correlatori\\
  Prof. Alberto Paoluzzi & Dott.~Enrico Marino, Dott.~Federico Spini\\
\end{tabularx}
}
\end{center}
\vskip 1.5cm \centerline{Anno Accademico 2014/2015} \normalsize
 %Parte2 Sperati

\newpage

\cleardoublepage
\addcontentsline{toc}{chapter}{Conclusioni e sviluppi futuri}   % aggiunge la bibliografia nell'indice

\chapter*{Conclusioni e sviluppi futuri}
\label{cha:conclusioni}

In conclusione...



\newpage



%% BibTex


%%%% PRE CITARE TUTTI  I RIFERIMENTI %%%%%
%\nocite{*}


%\bibliographystyle{unsrt}


\cleardoublepage
\addcontentsline{toc}{chapter}{Bibliografia}   % aggiunge la bibliografia nell'indice

\printbibliography

%% thebibliography
%\clearpage
\addcontentsline{toc}{chapter}{Bibliografia}
\begin{thebibliography}{9}
	\bibitem{lamport-latex} Leslie Lamport.
		\emph{\LaTeX: a document preparation system}. Addison-Wesley, 1994.
	
	\bibitem{companion} Michel Goossens, Frank Mittelbach, and Alexander Samarin. \emph{The \LaTeX\ Companion}.
		Addison-Wesley, 1994.
\end{thebibliography}

\end{document}