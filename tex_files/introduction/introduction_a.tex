Since the beginning of Internet, the ability to create and publish content on the web has made the success of Content Management Systems. Products like Joomla! or WordPress, born to handle simple websites or blogs, are evolved to support web applications of any sort (from personal portfolio to online shopping), running as of January 2015 more than 25\% of the top ten million websites \cite{cms_stats}. This evolution has been allowed by a plug-in based architecture, where each plug-in is responsible to handle a functionality subset of the whole application, presenting the user through a simple accessible configuration and management interface.
The large number of available plug-ins covers most of the common and frequently required customizations, thus avoiding to write ad-hoc code. Nevertheless, the implementation of specific functional characteristics inevitably requires to intervene at code level.

When the effort required to add custom features to a CMS results too expensive, a web framework can be adopted instead. A web framework consists of a set of software facilities that aims to alleviate the overhead associate with common development activities. Web application coding effort, while eased by the web framework, is anyway rewarded with an increased level of extensibility and customizability of the resulting application.
The most desirable features for a web framework are: user management, session management, database access via HTTP RESTful API. In order to effectively speed up web applications development, these facilities should be provided relying mostly on external configuration files and less on procedural code \cite{MIPRO}.

In this thesis a software platform named X-Project is introduced. It consists of a Web Component library, a development methodology and a set of guidelines.

The Web Components library is applied over a powerful web framework, Loopback by Strongloop, and realizes an hybrid prototypal tool which brings together the customizability of a modern web framework with the ease of use of traditional CMSs.

It has been naturally prompt by the use of the toolkit and, then, standardized a document-driven development process that polarizes the concept of reusing the code whose overall readability, maintainability and extendibility result dramatically increased.

Guidelines have been drafted in order to provide an aid to users to get the straight of the entire structure.

First goal of this project is the creation of a platform that can wrap the features of CMSs and web application frameworks. 
X-Project aims to simplify developers life in web applications creation and give them the chance to easily compose, via web components libraries, pages made of reused components.
Reusability and access to reusable components is an important feature that X-Project borrows from Web Components standard.
As further said in \ref{sec:XPR_xpr}, ``Everything is an element, even a service'' is the X-Project's philosophy: every kind of function or structure is encapsulated in Polymer elements, carrying to extreme levels web components concept. 
Moreover, using Single Page Application pattern, X-Project tries to streamline loading times of web applications.

The choice of cutting edge technologies, such as Polymer-Project by Google, has been done both to meet the challenge of new and unknown development pattern, and to test the newest technologies.

X-Project is located at the end of a natural path composed by user oriented CMSs and developer oriented web application framework (see \ref{sec:CMS_class}), trying to fuse CMSs ease of use with web application frameworks customization level.

The remainder of this document is divided in two parts and organized as follows. Part One counts of three chapters. Chapter One describes the state of the art of CMS's world, giving a classification and focusing on four of the major CMSs on the market. Chapter Two analyzes every used technology, describing the methodological approach of each one. Chapter Three provides an overview of Single Page Application development pattern, explaining pros and cons and technical functioning.
Part two is composed by four chapters. Chapter Four describes the project in his integrity, exposing the core of X-Project, its functionalities and its features.
Chapter Five focuses on a specific component that has been developed, Media Management Component with S3 service, and on all the theorical and practical concepts that there's behind the ideation of the component. Chapter Six presents a real use case with configuration steps and code snippets.
Finally, Chapter Six, exposes project conclusions and further implementations of the work.



