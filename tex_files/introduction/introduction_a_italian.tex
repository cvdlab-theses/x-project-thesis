Sin da le origini di Internet, la possibilità di creare e pubblicare contenuti sul web ha fatto sì che i Content Management Systems prendessero piede e avessero successo. Prodotti come Joomla! o WordPress, nati per gestire siti web o semplici blog, si sono evoluti per supportare applicazioni web di qualsiasi tipo (dal portfolio personale allo shopping on-line). Da statistiche di gennaio 2015 risulta che oltre il 25 \% dei primi dieci milioni di siti web è gestito con i maggiori CMS\cite{cms_stats}. Questa evoluzione è stata permessa da un architettura basata su plug-in, dove ogni plug-in è responsabile di gestire un sottoinsieme di funzionalità di tutta l'applicazione, presentandoso all'utente attraverso una semplice interfaccia di configurazione e una gestione estremamente accessibile.
Il gran numero di plug-in disponibile copre la maggior parte delle personalizzazioni più comunemente richieste, evitando così la scrittura di codice ad-hoc. Tuttavia, l'attuazione di specifiche caratteristiche funzionali richiede inevitabilmente l'intervento dello sviluppatore a livello di codice.