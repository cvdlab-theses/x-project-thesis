\section{S3 Component - Example}
\label{sec:S3_exmpl}

First example concerns the upload part of the component.
As shown before, \texttt{part-s3-upload} element calls an \texttt{api-s3-upload} element to handle the transfer from local computer to S3 cloud service.
In the HTML code of \texttt{part-s3-upload} element are implemented both the informations and input tags and the binding with the API.

\subsection{part-s3-upload}

\begin{lstlisting}[language=html]
<link rel="import" href="/components/api-s3-upload/api-s3-upload.html">

<dom-module id="part-s3-upload">
  <template>
    <api-s3-upload id="upload" folder="{{folder}}"
      file="{{file}}" file-name="{{fileName}}">
    </api-s3-upload>
    <input id="input" type="file" on-change="on_change">
  </template>
</dom-module>

<script>
  Polymer({
    on_change: function () {
      var file = this.$.input.files[0];
      if (!file) {
        return;
      }
      this.fileName = this.fileName || file.name;
      this.file = file;
      this.$.upload.send();
    }
  });
</script>

\end{lstlisting}

In this case, \texttt{api-s3-upload} the is triggered by the \texttt{on-change} function applied on the input box. As shown above \texttt{on-change} function checks the integrity of the file and save its name, then, using \$ selector that puts off to the tag with \texttt{id=``upload''}, triggers the API.


\subsection{part-list-item}

Second example concerns the retrieve and visualization part of the component.
As shown before, \texttt{part-list-item} is triggered by \texttt{part-list} element. \texttt{part-list-item}, in turn, triggers \texttt{api-s3-delete} API.
This element shows the image saved in S3 Bucket and its additional informations. Moreover \texttt{part-list-item} provides a button to delete the image from the bucket.

\begin{lstlisting}[language=html]
<link rel="import" href="/components/api-s3-delete/api-s3-delete.html">
<dom-module id="part-s3-list-item">
  <template>

    <api-s3-delete id="request" file_name="{{item.key}}"></api-s3-delete>

    <div id="image">
      <img id="thumb" src="{{item.url}}" preview$="{{preview}}">
      <template is="dom-if" if="{{preview}}">
        <ul id="image_labels">
          <li>Name: <span id="image_details">{{item.key}}</span></li>
          <li>Modified: <span id="image_details">{{item.lastmodified}}</span></li>
          <li>Size: <span id="image_details">{{item.size}}</span> KB</li>
        </ul>
        <button id="delete" on-click="on_click">Delete</button>
      </template>
    </div>

  </template>
</dom-module>
<script>
  Polymer({

    delete_image: function () {
      this.$.request.send();
    }
});
</script>
\end{lstlisting}


In this case, \texttt{api-s3-delete} the is triggered by the \texttt{on-click} function applied on the delete button. As shown above \texttt{on-click} function, using \$ selector that puts off to the tag with \texttt{id=``request''}, triggers the API.


\begin {figure}[h]
\graphicspath{{images/chapter_s3/}}
\includegraphics[width=\textwidth]{s3_example}
\caption{S3 Component example - Upload and preview functions}
\end {figure}

