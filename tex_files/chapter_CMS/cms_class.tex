\section{CMS classification}
\label{sec:CMS_class}

I CMS classici e più famosi (e.g. Joomla! e Wordpress) hanno quasi del tutto eliminato la figura dello sviluppatore, inserendo dei tool di creazione grafica e fornendo dei preset che coprono larga parte delle richieste più comuni. Altri CMS che si stanno affermando in questo momento hanno invece fatto dello sviluppatore un punto di forza, puntando molto sulla possibilità di personalizzazione e customizzazione del prodotto.

Since the beginning of Internet, the ability to create and publish content on the web has made the success of Content Management Systems. Products like Joomla! or WordPress, born to handle simple websites or blogs, are evolved to support web applications of any sort (from personal portfolio to on-line shopping), running as of January 2015 more than 25 percent of the top ten million websites \cite{cms_stats}. This evolution has been allowed by a plug-in based architecture, where each plug-in is responsible to handle a functionality subset of the whole application, presenting the user through a simple accessible configuration and management interface.
The large number of available plug-ins covers most of the common and frequently required customizations, thus avoiding to write ad-hoc code. Nevertheless, the implementation of specific functional characteristics inevitably requires to intervene at code level.

Negli ultimi anni, invece, sta prendendo forma un ritorno alle origini, ovvero si sta restituendo importanza alla figura dello sviluppatore. Cms come KeystoneJS e Ghost.org fanno della personalizzazione un punto di forza.
Dunque si favorisce la possibilità di customizzazione del prodotto, rispetto alla facilità d'uso e alla possibilità di utilizzo anche ai non addetti ai lavori. 

Si possono distinguere dunque due categorie di CMS:
\begin{itemize}

\item cms orientati all'utente, perlopiù visuali e preimpostati;
\item cms orientati allo sviluppatore, che necessitano di codice per poter essere utilizzati dall'amministratore.

\end{itemize}