\section{CMS classification}
\label{sec:CMS_class}

The classic and most popular CMS (e.g. Joomla! and Wordpress) have nearly eliminated the figure of the developer, inserting graphics creation tools and providing presets covering a large part of the most common requests. Other CMSs, that are emerging at this time, instead, have done of the developer a strength, heavily focusing on the possibility of personalization and customization of the product.

Since the beginning of Internet, the ability to create and publish content on the web has made the success of Content Management Systems. Products like Joomla! or WordPress, born to handle simple websites or blogs, are evolved to support web applications of any sort (from personal portfolio to on-line shopping), running as of January 2015 more than 25\% of the top ten million websites \cite{cms_stats}. This evolution has been allowed by a plug-in based architecture, where each plug-in is responsible to handle a functionality subset of the whole application, presenting the user through a simple accessible configuration and management interface.
The large number of available plug-ins covers most of the common and frequently required customizations, thus avoiding to write ad-hoc code. Nevertheless, the implementation of specific functional characteristics inevitably requires to intervene at code level.

In recent years, however, it is taking shape a return to basics. New CMSs are giving importance to the developer figure. CMSs as KeystoneJS and Ghost.org make of customization a strength.
Therefore favors the option of product customization, compared to the ease of use and the ability to availment even for non-experts.
Nowadays, two categories of CMSs can be distinguished:
\begin{itemize}
\item User oriented CMSs: mostly visual and presetted;
\item Developer oriented CMSs: customizable frameworks that need written code to be used by the admin.
\end{itemize}