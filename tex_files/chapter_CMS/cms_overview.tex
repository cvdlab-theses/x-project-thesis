\section{CMS overview}
\label{sec:CMS_overview}

This section consists of an overview about Content Management Systems.

A Content Management System is an application, that provides capabilities to multiple users with different levels of permission to manage content, data or information of a website project, or internet application \cite{cms_def}.

Managing content refers to creating, editing, archiving, publishing, collaborating on, reporting, distributing website content, data and information.
So, a CMS is an application that allows creating and publishing content from a central interface. CMSs are often used to run websites containing blogs, news, and shopping. Many corporate and marketing websites use CMSs. CMSs typically aim to avoid the need for hand coding, but may help it for specific elements or entire pages \cite{cms_over}. 

A Web CMS may catalog and index content, select or assemble content at runtime, or deliver content to specific visitors in a requested way, such as other languages. Web Content Management Systems usually allow client control over HyperText Markup Language - based content, files, documents, and Web hosting plans based on the system depth and the niche it serves.


CMSs world involves three main actors: the developer, the admin and the user. Each actor plays a fundamental role in the life of a CMS managed website. The developer is the one who physically creates the website, manages the functionality, designs the graphics and makes provide the Admin with set-up.
The administrator is not a technician (not a programmer) who manages the website content: he can create, edit, publish and administer the material to be shown.
He manages the contents of the site through an administration panel specifically created by the developer. The administrator works in a reserved area, only accessible via a login panel.
Finally, the user is the one who benefits of the contents made available by the admin. User cannot access the administration panel and, therefore, can not create, edit or delete content (sometimes users can comment on existing content).

\begin {figure}[h]
\graphicspath{{images/chapter_cms/}}
\includegraphics[width=\textwidth]{cms_schema}
\caption{CMS's actors schema}
\end {figure}


