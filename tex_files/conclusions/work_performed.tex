\section{Work performed}
\label{sec:conclusions_work_performed}

Gran parte del lavoro svolto in questa tesi è servito per mettere le radici al progetto X-Project. Come visto nei capitoli precendi il lavoro è basato su un concetto che unisce diverse tecnologie e servizi, il quale da vita ad un vero e proprio content managment system. 
Tale metodologia ha portato ad un prodotto funzionale e all'avanguardia, lo svilluppo del caso d'uso è servito per capire i veri vantaggi del funzionamento di X-Project e ad afferrare che può essere impiegato per lo sviluppo di molti modelli di web application.
I vantaggi principali della tesi posso essere divisi in due parti: lato client e lato server; nel lato client sicuramente il vantaggio più importante è nell'uso dei WebCompnents, i quali sono totalmente riusabili e semplici da usare. Un altro vantaggio che risiede nell'uso dei WebComponents è la struttura di questi ultimi, la quale è composta da parti totalmente indipendenti tra loro; le entità che compongono un elemento sono: la struttura, il comportamento e la presentazione.
Nel lato server l'utilizzo di LoopBack ha portato dei grossi vantaggi dati dalla facilità con cui si possono instanziare API semplicemnte dalla definizione del modello che si intende utilizzare.