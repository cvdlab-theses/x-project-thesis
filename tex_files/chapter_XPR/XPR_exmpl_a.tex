\section{User Management example: login}
\label{sec:XPR_exmpl}

The example is based on the user login service. 
For this function has been implemented both the client side and the backend. So for the client side has been developed the element \texttt{<form-login>}, for the server side has been developed the element \texttt{<api-user-login>}.

The elements that have been developed are:

\subsubsection{\texttt{<api-user-login>}}

Login a user with the given credentials.

\begin{lstlisting}[language=html]
<api-user-login credentials="{{credentials}}"
collection="{{collection}}" 
response="{{response}}" error="{{error}}"/>
\end{lstlisting}
Where:
\begin{itemize}
\item \texttt{credentials} email and password of user.
\item \texttt{collection} name of collection(Object).
\item \texttt{response}	HTTP response message(String).
\item \texttt{error} object of the error response(Object).
\end{itemize}

\subsubsection{\texttt{<form-login>}}

Create a login form.

\begin{lstlisting}[language=html]
<form id="form" on-submit="on_submit">
    <div class="field">
        <label class="label">email</label>
        <input class="input" is="iron-input" type="text" 
    		placeholder="email" 
            bind-value="{{credentials.email}}">
    </div>
    <div class="field">
        <label class="label">password</label>
        <input class="input" is="iron-input" type="password" 
        	placeholder="password" 
        	bind-value="{{credentials.password}}">
    </div>
      	<input type="submit" value="login"/>
</form>
\end{lstlisting}


\begin {figure}[h]
\graphicspath{{images/chapter_USR/}}
\includegraphics[width=\textwidth]{usr1}
\caption{Login Element Example}
\end {figure}