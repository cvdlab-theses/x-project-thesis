\section{X-Project}
\label{sec:XPR_xpr}

X-Project is a platform to build full-stack Javascript NodeJS API-centric HTML5 based Single Page Application with Web Components via Polymer-Project.
X-Project is composed by guidelines, methodology and a library of elements.
The document driven web development methodology and guidelines allow to build a very structured and usable Single Page Application.

``Everything is an element'' is the philosophy of the project.

The joint use of Web Components and Strongloop Loopback framework following the document driven web development methodology allowed to create vertical widget that influence every level of the stack. With a descriptive implementation it is possible to give life to API, on the server side, and visual and functional widget on the client side.

A Web Application is essentially built by composing elements together.

The goal of X-Porject is to allow to build a Web Application by composing existing elements. This assumption comes from web application's sharing of essential non-specific components.
Moreover X-Project's goal is to achieve the same easiness of adding an html element, adding logic and function elements.
In fact, in X-Project, an element is a part of the application that comprises both the client side and the server side.

X-Project was born to exploit Web Components benefits and extend them to extreme levels. In fact, X-Project, borrows all Web Components and Polymer-Project benefits and adds to them the ones that come from document driven development methodology.
Huge benefits that come from web components environment are reusability and access to reusable code. In fact, by creating stand alone vertical widgets, it's possibile to use components in various shapes.
The encapsulation, proper of web components, allows to reuse elements with no concerns of dependencies and specific behaviors.
Moreover, thanks to Web Components philosophy and X-Project guidelines, it's easy to gain a clear separation between structure, content, behavior and presentation of elements. 
In fact, it's possible to create components that concern the only presentation part of an element, such as mixins in which developers can express groups of CSS rules to be applied to different elements. This aspect also makes style extremly reusable.
In X-Project guidelines this pattern is applied to every kind of element.

Moreover, in the world of web application building platform, there are limits and gaps in terms of ease of use and orientation. As said in \ref{sec:CMS_class}, it is possbile to make a classification of these platforms based on orientation patterns. Nowadays, user oriented platforms have lacks when projects assume big dimensions mainly because of their lack of structure. Developer oriented platforms don't have the same limits, but are defective in ease of use. 

