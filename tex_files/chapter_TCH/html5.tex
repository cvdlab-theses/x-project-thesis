\section{HTML 5}
\label{sec:TCH_html5}

In this section there will be an overview about Html5.

HTML5 is the latest version of Hypertext Markup Language, the code that describes web pages. It's actually three kinds of code: HTML, which provides the structure; Cascading Style Sheets (CSS), which take care of presentation; and JavaScript, which makes things happen.

HTML5 has been designed to deliver almost everything you'd want to do online without requiring additional software such as browser plugins. It does everything from animation to apps, music to movies, and can also be used to build incredibly complicated applications that run in your browser.

There's more. HTML5 isn't proprietary, so you don't need to pay royalties to use it. It's also cross-platform, which means it doesn't care whether you're using a tablet or a smartphone, a netbook, notebook or ultrabook or a Smart TV: if your browser supports HTML5, it should work flawlessly. Inevitably, it's a bit more complicated than that. More about that in a moment.

While some features of HTML5 are often compared to Adobe Flash, the two technologies are very different. Both include features for playing audio and video within web pages, and for using Scalable Vector Graphics. HTML5 on its own cannot be used for animation or interactivity – it must be supplemented with CSS3 or JavaScript. There are many Flash capabilities that have no direct counterpart in HTML5. See Comparison of HTML5 and Flash.

Although HTML5 has been well known among web developers for years, its interactive capabilities became a topic of mainstream media around April 2010 after Apple Inc's then-CEO Steve Jobs issued a public letter titled Thoughts on Flash where he concluded that ``Flash is no longer necessary to watch video or consume any kind of web content'' and that ``new open standards created in the mobile era, such as HTML5, will win''.This sparked a debate in web development circles where some suggested that while HTML5 provides enhanced functionality, developers must consider the varying browser support of the different parts of the standard as well as other functionality differences between HTML5 and Flash. In early November 2011, Adobe announced that it would discontinue development of Flash for mobile devices and reorient its efforts in developing tools using HTML5.
